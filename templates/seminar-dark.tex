% ==========================================================
% 1. PENGATURAN BAHASA & ENCODING
% ==========================================================
\usepackage[bahasa]{babel}      % Penyesuaian format tanggal/nama Indonesia
\usepackage[utf8]{inputenc}     % Encoding karakter standar
\usepackage[T1]{fontenc}        % Encoding font output
\usepackage{physics}

\usepackage{booktabs}
\usepackage[table]{xcolor}
\usepackage{graphicx}

% ==========================================================
% 2. PENGATURAN FONT (Comic Style + Math Euler)
% ==========================================================
% --- Font Teks ---
\usepackage[scaled=1.2]{comicneue}  
\renewcommand{\familydefault}{\sfdefault}
\usefonttheme{structurebold}

% --- Font Matematika ---
\usefonttheme[onlymath]{serif}     % Matematika serif
\usepackage{mathpazo}              % Palatino untuk operator
\usepackage{eulervm}               % Euler untuk variabel 

% FIX: Sembunyikan warning bahwa hbar diganti hslash
\usepackage{silence}
\WarningFilter{eulervm}{Symbol \hbar not available}
\usepackage{eulervm}

% --- TEKNIK PENEBALAN TEKS (MICRO-STROKE) ---
% Menambahkan garis tepi tipis agar font terlihat lebih tebal
% Sesuaikan LineWidth: 0.1pt (Tipis) s.d 0.4pt (Tebal)
\usepackage{pdfrender}
\AtBeginDocument{
    \pdfrender{
        TextRenderingMode=FillStroke,
        LineWidth=0.2pt,  % <--- Ubah untuk atur ketebalan
        FillColor=.,      % Warna isi mengikuti teks saat ini
        StrokeColor=.     % Warna garis tepi mengikuti teks saat ini
    }
}

% --- Pengaturan Vektor (Gaya Papan Tulis) ---
%\usepackage[d]{esvect}             
%\renewcommand{\vec}{\vv}
\usepackage{bbm} 
\renewcommand{\mathbf}[1]{\mathbbm{#1}}

% ==========================================================
% 3. PAKET MATEMATIKA & ALGORITMA
% ==========================================================
\usepackage{amsmath,amssymb}
\usepackage{algorithm}         
\usepackage{algpseudocode}

% ==========================================================
% 4. PENGATURAN SPASI & LAYOUT TEKS
% ==========================================================
\usepackage{xpatch}             

% Mengatur jarak antar paragraf
\setlength{\parskip}{0.5em}

% Patch untuk memberi jarak antar item list (bullet points)
% Itemize (Bullet) -> 0.4em
\xpatchcmd{\itemize}{\def\makelabel}{\setlength{\itemsep}{0.4em}\def\makelabel}{}{}
% Enumerate (Angka) -> 0.6em
\xpatchcmd{\enumerate}{\def\makelabel}{\setlength{\itemsep}{0.6em}\def\makelabel}{}{}

% Margin Kiri/Kanan Slide
\setbeamersize{text margin left=1cm, text margin right=1cm}

% ==========================================================
% 5. GRAFIK, WARNA & TIKZ
% ==========================================================
\usepackage{graphicx}
\usepackage{pdfpages} 
\usepackage{tikz, pgfplots}
\usetikzlibrary{shapes, arrows.meta, positioning, calc, backgrounds, fit, shadows, decorations.pathmorphing}
\pgfplotsset{compat=1.18}
\tikzset{wave/.style={decorate, decoration={snake, amplitude=2pt, segment length=8pt}}}

% --- Definisi Palet Warna (Chalk/Kapur) ---
\definecolor{cb}{RGB}{85, 160, 220}   % Chalk Blue
\definecolor{cr}{RGB}{219, 112, 147}  % Chalk Red 
\definecolor{co}{RGB}{255, 170, 80}   % Chalk Orange
\definecolor{cg}{RGB}{140, 220, 140}  % Chalk Green 
\definecolor{cw}{RGB}{240, 240, 240}  % Chalk White 
\definecolor{BoardDeep}{HTML}{111111} % Background Gelap

% --- FIX TIKZ PICTURE (TOTAL RESET) ---
% Perintah ini akan mematikan efek "Fake Bold" khusus TikZ

\tikzset{
    every picture/.append style={
        execute at begin picture={
            % Reset mode ke "Fill" saja (tanpa garis tepi)
            \pdfrender{TextRenderingMode=Fill, LineWidth=0pt}
        }
    }
}

% ==========================================================
% 6. PENGATURAN KODE PROGRAM (MINTED)
% ==========================================================
\usepackage{minted}
\usemintedstyle{monokai}

\setminted[python]{
    fontsize=\footnotesize,   
    linenos=false,          
    breaklines=true,        
    tabsize=4,
    frame=none,             
    bgcolor={},             % Transparan (menyatu bg slide)
}

% ==========================================================
% 7. TEMA BEAMER: WARNA ELEMEN
% ==========================================================
% --- Warna Latar & Teks Utama ---
%\setbeamercolor{background canvas}{bg=BoardDeep} 

% Menggunakan file 'chalkboard.jpg' sebagai latar belakang
\usebackgroundtemplate{%
\tikz[overlay,remember picture] \node[opacity=1, at=(current page.center)] {
   % Pastikan file chalkboard.jpg ada di folder yang sama!
   \includegraphics[height=\paperheight, 
   width=\paperwidth]{templates/chalkboard.jpg}};
}
\setbeamercolor{normal text}{fg=cw}      

% --- Warna Header & Footer ---
\setbeamercolor{frametitle}{fg=white}
\setbeamercolor{title}{fg=cw}
\setbeamercolor{subtitle}{fg=cb}
\setbeamercolor{author}{fg=white!70!blue}
\setbeamercolor{institute}{fg=white!70!gray}
\setbeamercolor{date}{fg=white!70!gray}

% --- Warna Blok & Caption ---
\setbeamertemplate{blocks}[rounded][shadow=false]
\setbeamercolor{block title}{fg=cg, bg=black!85}
\setbeamercolor{block body}{fg=white, bg=black!80}
\setbeamercolor{block title alerted}{fg=cr, bg=black!60}
\setbeamercolor{block body alerted}{fg=cw, bg=black!30}
\setbeamercolor{block title example}{fg=co, bg=black!60}
\setbeamercolor{block body example}{fg=cw, bg=black!30}

% --- Caption ---
\setbeamercolor{caption}{fg=cg}

% --- Warna List (Itemize/Enumerate) ---
\setbeamertemplate{itemize items}{\color{co}$\blacktriangleright$} % Bullet Segitiga Oranye
\setbeamercolor{enumerate item}{fg=cb}        % Angka Biru
\setbeamercolor{enumerate subitem}{fg=cb}
\setbeamercolor{enumerate subsubitem}{fg=cb}

% ==========================================================
% 8. TEMA BEAMER: DAFTAR ISI (TOC)
% ==========================================================
% Menggunakan warna ungu manual (purple) untuk panah di TOC
\setbeamercolor{section in toc}{fg=cw}
\setbeamercolor{subsection in toc}{fg=cb}
\setbeamercolor{subsubsection in toc}{fg=white!50!gray}
\setbeamercolor{section number projected}{bg=blue!80, fg=black}
\setbeamercolor{toc title}{fg=yellow!80!white}

\setbeamertemplate{section in toc}{%
  \leavevmode
  \textcolor{cr}{$\bullet$}\hspace{1em}%
  \inserttocsection\par}

\setbeamertemplate{subsection in toc}{%
  \hspace{1.5em}%
  \textcolor{purple}{$\blacktriangleright$}\hspace{1em}%
  \inserttocsubsection\par}

% ==========================================================
% 9. TEMA BEAMER: TEMPLATE HEADER & TITLE
% ==========================================================
% Hilangkan simbol navigasi kecil
\setbeamertemplate{navigation symbols}{} 

% Format Judul Frame (Header)
\setbeamertemplate{frametitle}{
  \nointerlineskip
  \begin{beamercolorbox}[
    wd=\paperwidth,    
    sep=1cm,           
    leftskip=0cm,      
    rightskip=0.5cm    
  ]{frametitle}
    \usebeamerfont{frametitle}\insertframetitle
  \end{beamercolorbox}
  \vspace*{-1.5em} 
}

% Format Caption Gambar
\setbeamertemplate{caption}{
  \footnotesize
  {\usebeamercolor[fg]{caption}\insertcaption}\par
}

% Format Halaman Judul (Title Page)
\defbeamertemplate*{title page}{centered logo}{
  \begin{center}
    \vspace*{0.15\paperheight} % Slightly adjusted top margin  
    % Title
    {\usebeamerfont{title}\inserttitle\par}    
    % Subtitle
    \vspace{0.8ex}
    {\usebeamerfont{subtitle}\insertsubtitle\par}    
    % Author (Added)
    \vspace{1em}
    {\usebeamerfont{author}\insertauthor\par}
    % Institute (Added)
    \vspace{0.5em}{\usebeamerfont{institute}\footnotesize\insertinstitute\par}    
    \vfill    
    % Logo/Graphic
    {\usebeamerfont{institute}\inserttitlegraphic\par}
    \vspace*{-2em}
  \end{center}
}

\setbeamertemplate{title page}[centered logo]
% ==========================================================
% 10. CUSTOM COMMANDS (MACRO)
% ==========================================================
% Macro untuk menyisipkan gambar dengan bayangan
\newcommand{\shadowfig}[3][]{%
  \tikz\node[inner sep=0pt, drop shadow, #1]{\includegraphics[#2]{#3}};%
}

% Shortcut untuk transisi fade
\newcommand{\fade}{\transfade}

% Symbol for enter:
\newcommand{\enterRight}{\reflectbox{\ensuremath{\hookleftarrow}}\,}