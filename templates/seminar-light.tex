% ==========================================================
% 1. PENGATURAN BAHASA & ENCODING
% ==========================================================
\usepackage[bahasa]{babel}      
\usepackage[utf8]{inputenc}     
\usepackage[T1]{fontenc}        

% ==========================================================
% 2. PENGATURAN FONT (Comic Style + Math Euler)
% ==========================================================
% --- Font Teks ---
\usepackage[scaled=0.9]{helvet}  
\renewcommand{\familydefault}{\sfdefault}
\usefonttheme{structurebold}

% --- Font Matematika ---
\usefonttheme[onlymath]{serif}      
\usepackage{mathpazo}               
\usepackage{eulervm}                

% FIX: Sembunyikan warning hbar
\usepackage{silence}
\WarningFilter{eulervm}{Symbol \hbar not available}

% ==========================================================
% 3. PAKET MATEMATIKA & ALGORITMA
% ==========================================================
\usepackage{amsmath,amssymb}
\usepackage{algorithm}          
\usepackage{algpseudocode}
\usepackage{colortbl} 

% ==========================================================
% 4. PENGATURAN SPASI & LAYOUT
% ==========================================================
\usepackage{xpatch}             
\setlength{\parskip}{0.5em}
\xpatchcmd{\itemize}{\def\makelabel}{\setlength{\itemsep}{0.4em}\def\makelabel}{}{}
\xpatchcmd{\enumerate}{\def\makelabel}{\setlength{\itemsep}{0.6em}\def\makelabel}{}{}
\setbeamersize{text margin left=1cm, text margin right=1.5cm}

% ==========================================================
% 5. GRAFIK, WARNA & TIKZ
% ==========================================================
\usepackage{graphicx}
\usepackage{pdfpages} 
\usepackage{tikz, pgfplots}
\usetikzlibrary{shapes, arrows.meta, positioning, calc, backgrounds, fit, shadows, decorations.pathmorphing}
\pgfplotsset{compat=1.18}
\tikzset{wave/.style={decorate, decoration={snake, amplitude=2pt, segment length=8pt}}}

% --- Definisi Palet Warna (Marker/Spidol) ---
\definecolor{cb}{RGB}{0, 80, 180}     % Marker Blue 
\definecolor{cr}{RGB}{200, 40, 40}    % Marker Red 
\definecolor{co}{RGB}{230, 100, 0}    % Marker Orange 
\definecolor{cg}{RGB}{0, 120, 60}     % Marker Green 
\definecolor{cp}{RGB}{110, 0, 110}    % Marker Purple 
\definecolor{cw}{RGB}{40, 40, 40}     % Tinta Hitam
\definecolor{BoardDeep}{HTML}{FFFFFF} % Background Putih
\definecolor{fr}{RGB}{224, 17, 95}    % Footer Red

% ==========================================================
% 6. PENGATURAN KODE PROGRAM (MINTED)
% ==========================================================
\usepackage{minted}
\usemintedstyle{tango} 

% Patch warna komentar (Menggunakan Oranye 'co')
\makeatletter
\AtBeginEnvironment{minted}{%
    \expandafter\def\csname PYGtango@tok@c\endcsname{\textcolor{co}}% 
    \expandafter\def\csname PYGtango@tok@c1\endcsname{\textcolor{co}}% 
    \expandafter\def\csname PYGtango@tok@cm\endcsname{\textcolor{co}}% 
    \expandafter\def\csname PYGtango@tok@cp\endcsname{\textcolor{co}}% 
}
\makeatother

\setminted[python]{
    fontsize=\footnotesize,    
    linenos=false,            
    breaklines=true,          
    tabsize=4,
    frame=lines,              
    rulecolor=\color{gray!50},
    bgcolor={},               
}

% ==========================================================
% 7. TEMA BEAMER: WARNA ELEMEN (WHITEBOARD)
% ==========================================================
\setbeamercolor{background canvas}{bg=BoardDeep} 
\setbeamercolor{normal text}{fg=cw}        

% --- Header & Footer ---
\setbeamercolor{frametitle}{fg=cp}        % Judul Frame jadi UNGU (cp)
\setbeamercolor{title}{fg=cp}             % Judul Utama jadi UNGU (cp)
\setbeamercolor{subtitle}{fg=co}          
\setbeamercolor{author}{fg=cw}
\setbeamercolor{institute}{fg=gray}
\setbeamercolor{date}{fg=gray}

% --- Warna Blok ---
\setbeamertemplate{blocks}[rounded][shadow=false]

% Block Standar (Hijau Muda)
\setbeamercolor{block title}{fg=cg, bg=cg!20}
\setbeamercolor{block body}{fg=cw, bg=cg!10}

% Alert Block (Merah Muda)
\setbeamercolor{block title alerted}{fg=cr, bg=cr!20}
\setbeamercolor{block body alerted}{fg=cw, bg=cr!10}

% Example Block (Oranye Muda)
\setbeamercolor{block title example}{fg=co, bg=co!20}
\setbeamercolor{block body example}{fg=cw, bg=co!10}

% --- Caption ---
\setbeamercolor{caption}{fg=cg}

% --- List Items ---
\setbeamertemplate{itemize items}{\color{co}$\blacktriangleright$}

% Custom Enumerate Item (Red Circle with White Number)
\setbeamertemplate{enumerate item}{%
    \tikz[baseline={([yshift=-.7ex]current bounding box.center)}]{%
        \node[
            circle,
            fill=cr,          % Fill with Marker Red
            text=white,       % Text White
            font=\bfseries\footnotesize,
            minimum size=1.4em,
            inner sep=0pt
        ] {\insertenumlabel};%
    }%
}

% Keep sub-items simple (Purple)
\setbeamercolor{enumerate subitem}{fg=cp} 
\setbeamercolor{enumerate subsubitem}{fg=cp}

% --- FIX: Add spacing before blocks to match text \parskip ---
\addtobeamertemplate{block begin}{\vspace{0.5em}}{}
\addtobeamertemplate{block alerted begin}{\vspace{0.5em}}{}
\addtobeamertemplate{block example begin}{\vspace{0.5em}}{}

% ==========================================================
% 8. TEMA BEAMER: DAFTAR ISI (TOC) - RAPAT
% ==========================================================
\setbeamercolor{section in toc}{fg=cw}        
\setbeamercolor{subsection in toc}{fg=cw}
\setbeamercolor{subsubsection in toc}{fg=gray}
\setbeamercolor{section number projected}{bg=cp, fg=white}
\setbeamercolor{toc title}{fg=cp}

% --- SECTION (Level 1) ---
\setbeamertemplate{section in toc}{%
  \leavevmode
  \textcolor{cr}{$\bullet$}\hspace{1em}%
  \inserttocsection
  \par
  % Adjusted to prevent overlap with subsections
  \vspace{-0.5em} 
}

% --- SUBSECTION (Level 2) ---
\setbeamertemplate{subsection in toc}{%
  \hspace{1.5em}%
  \textcolor{cp}{$\blacktriangleright$}\hspace{1em}%
  \inserttocsubsection
  \par
  % Reduced vertical space for subsections
  \vspace{-1em} 
}

% ==========================================================
% 9. HEADER, FOOTER & TITLE PAGE (UPDATED)
% ==========================================================
\setbeamertemplate{navigation symbols}{} 

% --- Define Colors for Split Footer ---
\setbeamercolor{footline_left}{bg=fr, fg=white}  % Left = Red
\setbeamercolor{footline_right}{bg=cw, fg=white} % Right = Dark Grey

% --- 1. NEW FOOTER (Split: Red Left, Black Right) ---
\setbeamertemplate{footline}{%
  \leavevmode%
  \hbox{%
    % LEFT HALF: Red Bar (Empty or can hold Author name)
    \begin{beamercolorbox}[wd=0.5\paperwidth, ht=2.5ex, dp=1.125ex, leftskip=1cm]{footline_left}%
      % \insertshortauthor % Uncomment if you want text here
    \end{beamercolorbox}%
    %
    % RIGHT HALF: Black Bar (Page Number)
    \begin{beamercolorbox}[wd=0.5\paperwidth, ht=2.5ex, dp=1.125ex, rightskip=2em]{footline_right}%
      \hfill 
      \tiny 
      \insertframenumber~/~\inserttotalframenumber
    \end{beamercolorbox}%
  }%
}

% --- 2. HEADER (Adjusted Spacing) ---
\setbeamertemplate{frametitle}{
  \nointerlineskip

% --- LOGO INSERTION START ---
  % This places the logo at the absolute top-right of the page
  \begin{tikzpicture}[remember picture, overlay]
      \node[anchor=north east, yshift=-0.2cm, xshift=-0.2cm] at (current page.north east) {
          % REPLACE 'example-image' WITH YOUR LOGO FILENAME
          \includegraphics[height=1cm]{templates/brinq-logo.png} 
      };
  \end{tikzpicture}
  % --- LOGO INSERTION END ---
  
  % --- Box Judul ---
  \begin{beamercolorbox}[
    wd=\paperwidth,     
    sep=1cm,            
    leftskip=0cm,       
    rightskip=0.5cm     
  ]{frametitle}
    \usebeamerfont{frametitle}\insertframetitle
  \end{beamercolorbox}
  \vspace*{-2em} 
}

\setbeamertemplate{caption}{
  \footnotesize
  {\usebeamercolor[fg]{caption}\insertcaption}\par
}

% --- 3. TITLE PAGE ---
\defbeamertemplate*{title page}{centered logo}{
  \begin{center}
    \vspace*{0.2\paperheight}
    {\usebeamerfont{title}\inserttitle\par}
    \vspace{0.8ex}
    {\usebeamerfont{subtitle}\insertsubtitle\par}
    \vfill
    {\usebeamerfont{institute}\inserttitlegraphic\par}
    \vspace*{-5em}
  \end{center}
}
\setbeamertemplate{title page}[centered logo]

% ==========================================================
% 10. CUSTOM MACROS (FIX PAGE 1)
% ==========================================================
\newcommand{\shadowfig}[3][]{%
  \tikz\node[inner sep=0pt, drop shadow, #1]{\includegraphics[#2]{#3}};%
}
\newcommand{\fade}{\transfade}
\newcommand{\enterRight}{\reflectbox{\ensuremath{\hookleftarrow}}\,}

% --- MACRO COVER BARU (Updated for Split Footer) ---
\newcommand{\printcover}{
    {
    % 1. Hide Header
    \setbeamertemplate{headline}{} 
    \setbeamertemplate{frametitle}{} 
    
    % 2. Custom Footer for Cover: Split Bar, but NO Numbers
    \setbeamertemplate{footline}{%
      \leavevmode%
      \hbox{%
        % Left Half: Red
        \begin{beamercolorbox}[wd=0.5\paperwidth, ht=2.5ex, dp=1.125ex]{footline_left}%
        \end{beamercolorbox}%
        % Right Half: Black (Empty)
        \begin{beamercolorbox}[wd=0.5\paperwidth, ht=2.5ex, dp=1.125ex]{footline_right}%
        \end{beamercolorbox}%
      }%
    }

    % 3. Render Title Page
    \begin{frame} 
        \titlepage
    \end{frame}
    }
    % 4. Reset page count
    \setcounter{framenumber}{0} 
}